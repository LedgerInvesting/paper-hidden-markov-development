Loss development modelling is the actuarial practice of predicting the 
total \textit{ultimate} losses incurred on a set of policies once all 
claims are reported and settled. This poses a challenging prediction 
task as losses frequently take years to fully emerge from 
reported claims, and not all claims might yet be reported.
Loss development models frequently estimate a set of 
\textit{link ratios} from insurance loss triangles, which are multiplicative 
factors transforming losses at one time point to ultimate. However, 
link ratios estimated using classical methods typically underestimate 
ultimate losses and cannot be extrapolated outside the domains of the 
triangle, requiring extension by \textit{tail factors} from another model. 
Although flexible, this two-step process relies on
subjective decision points that might bias inference.
Methods that jointly estimate `body’ link ratios and 
smooth tail factors offer an attractive 
alternative. This paper proposes a novel application of Bayesian 
hidden Markov models to loss development modelling, where discrete, 
latent states representing body and tail processes are automatically 
learned from the data. The hidden Markov development model 
is found to perform comparably to, and frequently better than, the 
two-step approach on numerical examples and industry datasets.

\textit{Keywords}: actuarial science, loss reserving, mixture modelling, time series
