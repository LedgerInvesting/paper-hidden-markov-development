The hidden Markov development model is compared to
$i$) a slightly more parsimonious latent change-point
model and $ii$) a two-step modelling approach traditional
in actuarial science.
Denote $\tau \in \{2, ..., M\}$ the body to tail
switch-over development period, and
$\bm{\rho} = (\rho_{1}, \rho_{2}) \in \{2, ..., M\}$,
where $\rho_{1} < \rho_{2}$,
a vector of tail start and end training window
development points, respectively.
While these constants could in theory vary over
experience periods, there is typically insufficient
data to do so.

In the traditional two-step modelling process,
the chain ladder
method is first fit to training data up
to development period $\tau - 1$ and
predictions of the lower diagonal loss
triangle are made up to $\tau$, only.
Secondly, the chosen tail model is fit to data
lying within the development period
interval $[\rho_{1}, \rho_{2}]$,
and predictions from $\tau$ made to 
some arbitrary development period, $j^{*}$.
The challenge and art of this two-step process
is in finding a value for $\tau$ that
identifies the development period where
losses are plateauing in the tail,
and finding values for $\bm{\rho}$ that
identify a suitable decaying curve of link ratios.
While $\tau = \rho_{1}$ in some cases,
more generally $\rho_{1} \leq \tau$.
The two-step approach differs from
equation (\ref{eq:hmm}) in only two
ways:

\begin{align}
\begin{split}
	y_{ij} &\sim 
	\begin{cases}
		\mathrm{Lognormal}(\mu_{1}, \sigma_{ij}) \quad j < \tau\\
		\mathrm{Lognormal}(\mu_{2}, \sigma_{ij}) \quad \rho_{1} \leq j \leq \rho_{2}
	\end{cases}\\
	\log \bm{\alpha}_{1:\tau - 2} &\sim \mathrm{Normal}(0, 1)\\
\end{split}
\end{align}
%
where now the decision between the two
models is decided by $\tau$ and $\bm{\rho}$.
The further exception in the two-step approach
is the use of a standard normal prior
on the log-scale link ratios, rather than the
regularising prior used in the hidden Markov models,
because non-identifiability is, in general,
no longer a concern.

The latent change-point model offers a compromise
between the two-step modelling process
and the hidden Markov model.
Like the hidden Markov model, the chain ladder
and generalized Bondy models are fit jointly, 
and like the two-step process the models
switch over at some development period
$\tau$. However,
$\tau$ is considered to be a
discrete parameter estimated
in the model directly.
The $\bm{\rho}$ parameters are not necessary
because the generalized Bondy process is
only fit from $\tau$ onwards.
In practice, due to its discrete nature,
$\tau$ is marginalised out of the model,
and the prior on $\tau$, $p(\tau)$,
is a discrete uniform distribution over the possible
development period switch-over points in $\{2,...,M\}$.

\begin{align}
\begin{split}
	y_{ij} &\sim 
	\begin{cases}
		\mathrm{Lognormal}(\mu_{1}, \sigma_{ij}) \quad j < \tau\\
		\mathrm{Lognormal}(\mu_{2}, \sigma_{ij}) \quad j \geq \tau\\
	\end{cases}\\
    \log \bm{\alpha}_{1:M - 1} &\sim \mathrm{Normal}(0, \scriptstyle{\frac{1}{1:M-1}})\\
    \tau &\sim \mathrm{DiscreteUniform}(2, M)\\
\end{split}
\end{align}
%
Like the hidden Markov model, a regularising
prior was placed on the log-scale link ratios of the chain-ladder
component to handle potential non-identifiability
at higher development periods where the process
is more likely to be in the tail.
